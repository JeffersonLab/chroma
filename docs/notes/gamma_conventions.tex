\documentclass[12pt]{article}
\usepackage{amsmath}
\newcommand{\ch}{{\rm DR}}
\newcommand{\dirac}{{\rm Di}}
\newcommand{\ps}{{\rm PS}}
%\pagestyle{empty}
\begin{document}
\begin{center}
{\bfseries\Large Summary of Gamma Matrix conventions}\\[1.0ex]
\end{center}
\setcounter{section}{1} In this document we list the gamma matrix
conventions employed by the various groups within LHPC, and the
similarity transformations between them; the document is essentially a
summary of Urs Heller's notes, which additional discussion of the
propagator calculation.  In each case the basic \textit{Euclidean}
gamma matrices satisfy
\begin{gather*}
\{ \gamma_{\mu}, \gamma_{\nu} \} = 2 \delta_{\mu\nu}, \qquad 
\gamma_\mu^{\dagger} = \gamma_\mu \\
\sigma_{\mu\nu} = \frac{i}{2}[\gamma_\mu,\gamma_\nu] \\
\Sigma_i = -\frac{1}{2}\epsilon_{ijk}\sigma_{jk} = -i\gamma_5\gamma_4\gamma_i\quad , \quad
[\Sigma_i,\Sigma_j] = 2 i \epsilon_{ijk}\Sigma_k \ .
\end{gather*}
Charge conjugation satisfies
\begin{gather*}
-{\rm C} = {\rm C}^{\rm T} = {\rm C}^{-1} = {\rm C}^\dagger \\
{\rm C} \gamma_\mu {\rm C}^{-1} = -\gamma_\mu^{T}, \quad
{\rm C} \sigma_{\mu\nu} {\rm C}^{-1} = -\sigma_{\mu\nu}^{T}, \quad
{\rm C} \gamma_5 {\rm C}^{-1} = \gamma_5^{T}, \quad
{\rm C} \gamma_5\gamma_\mu {\rm C}^{-1} = (\gamma_5\gamma_\mu)^{T}\ .
\end{gather*}
In order to construct the similarity transformation between different
representations, only the representation.
However there are differences between the different groups in the
definitions of the ``derived'' gamma matrices.
% in particular as to whether they are all Hermititian. 

Note that the Pauli spin matrices are
\begin{displaymath}
\sigma_1 = \begin{pmatrix} 0 & 1\\ 1 & 0 \end{pmatrix}, \quad
\sigma_2 = \begin{pmatrix} 0 & -i\\ i & 0 \end{pmatrix}, \quad
\sigma_3 = \begin{pmatrix} 1 & 0\\ 0 & -1 \end{pmatrix}.
\end{displaymath}

%\end{eqnarray}
%\end{equation}

\subsection{Gamma matrix basis}
\subsubsection{UKQCD (twisted Dirac)}
This is basically the Dirac convention, defined by
\begin{gather*}
\vec{\gamma} = i \begin{pmatrix}0 & \vec{\sigma}\\ -\vec{\sigma} & 0 \end{pmatrix},\quad
\gamma_4 = \begin{pmatrix} {\rm I} & 0\\ 0 & -{\rm I} \end{pmatrix} \\
{\rm C} = \gamma_2\gamma_4 = \begin{pmatrix} 0 & -i\sigma_2\\ -i\sigma_2 & 0 \end{pmatrix}
\end{gather*}
In the following, this basis will be denoted $\Gamma_\dirac$.

\subsubsection{MIT (Dirac)}
The same as UKQCD, but with the sign of the spatial components reversed
\begin{gather*}
\vec{\gamma} = -i\begin{pmatrix} 0 & \vec{\sigma}\\ -\vec{\sigma} & 0 \end{pmatrix},\quad
\gamma_4 = \begin{pmatrix} {\rm I} & 0\\ 0 & -{\rm I} \end{pmatrix} \\
{\rm C} = \gamma_2\gamma_4 = \begin{pmatrix} 0 & i\sigma_2\\ i\sigma_2 & 0\end{pmatrix}
\end{gather*}
The conventions here are the same as Minkowski Dirac with the substitution
$\gamma_k^{(E)} = -i\gamma^{(M) k}$, $k=1,2,3$.

\subsubsection{SZIN}

This is the Degrand-Rossi (DR) basis, a variation of a chiral basis, i.e.\
in which $\gamma_5$ is diagonal, but with a different sign for
$\gamma_2$.  It is used by
MILC and pretty will the remainder of the US community:
\begin{gather*}
\gamma_1 = \begin{pmatrix} 0 & i \sigma_1 \nonumber \\ -i \sigma_1 & 0 \end{pmatrix},\quad
\gamma_2 = \begin{pmatrix} 0 & -i \sigma_2 \\ i \sigma_2 & 0 \end{pmatrix},\quad
\gamma_3 = \begin{pmatrix} 0 & i \sigma_3 \\ -i \sigma_3 & 0 \end{pmatrix},\quad
\gamma_4 = \begin{pmatrix} 0 & I\\ I & 0 \end{pmatrix} \\
{\rm C} = \gamma_2\gamma_4 = \begin{pmatrix} -i\sigma_2 & 0\\ 0 & i\sigma_2 \end{pmatrix}
\end{gather*}
Note the sign in $\gamma_2$.  Matrices in this basis will be denoted
by $\Gamma_\ch$.

\subsubsection{Kentucky-Adelaide}
This is a Euclidean version of the Pauli-Schwinger basis used by 
Sakurai:
\begin{gather*}
\vec{\gamma} = \begin{pmatrix} 0 & \vec{\sigma}\\ \vec{\sigma} & 0 \end{pmatrix}, \quad
\gamma_4 = \begin{pmatrix} {\rm I} & 0\\ 0 & -{\rm I} \end{pmatrix}, \quad
\gamma_5 = i \begin{pmatrix} 0 & -{\rm I}\\ {\rm I} & 0 \end{pmatrix} \\
{\rm C} = \gamma_2\gamma_4 = \begin{pmatrix} \sigma_2 & 0\\ 0 & -\sigma_2 \end{pmatrix}
\end{gather*}
In the following, this basis will be denoted $\Gamma_\ps$. NOTE: in papers,
these groups claim to use an opposite sign for $\sigma_{\mu\nu}$ than written
at the beginning.

\subsection{Similarity Transformation}
Here we give the unitary matrix that relates the SZIN basis to the
UKQCD Dirac basis, from Urs Heller's notes.
\[
\Gamma_\ch = U^{\dagger} \Gamma_\dirac U
\]
where the similarity transformation is realised through
\begin{eqnarray}
U & = &\frac{1}{\sqrt{2}} \left(
\begin{array}{rr}
i \sigma_2 & i \sigma_2\\
i \sigma_2 & -i \sigma_2
\end{array}
\right)\nonumber \\
& \equiv & \frac{1}{\sqrt{2}}
\left(
\begin{array}{rrrr}
 0 & 1 & 0 & 1\\
 -1 & 0 & -1 & 0\\
 0 & 1 & 0 & -1\\
-1 & 0 & 1 & 0
\end{array}
\right).\label{eq:sim_def}
\end{eqnarray}
Note that $U^{\dagger} = U^T = -U$.

\subsection{Propagator projection operators}
We now discuss how to use use the projection property to compute the
non-relativistic quark propagators in the DR basis.

The non-relativistic projection consists of solving the propagator
equation
\begin{equation}
M G = S \frac{1}{2} (1 + \gamma_4)
\end{equation}
where $S$ is some source matrix.  In a Dirac basis, the projection
operator
\begin{equation}
\frac{1}{2} (1 + \gamma_4) = \left( 
\begin{array}{rr}
I_2 & 0 \\
0 & 0
\end{array}
\right)
\end{equation}
and we have
\begin{equation}
M_\dirac G_\dirac = S_\dirac \frac{1}{2} (1 + \gamma_{4,\dirac}).
\label{eq:prop_dirac}
\end{equation}
In the chiral basis we have
\begin{equation}
M_\ch G_\ch = S_\ch \frac{1}{2} (1 + \gamma_{4,\ch}).
\label{eq:prop_chiral}
\end{equation}
We now perform the similarity transformation on the projection
operator to yield
\begin{equation}
M_\ch G_\ch = S \frac{1}{2} U^{\dagger} (1 + \gamma_{4,\dirac}) U.
\end{equation}
Multiplying this equation on the right by $U^{\dagger}$ yields
\begin{equation}
M_\ch G_\ch U^{\dagger} = S U^{\dagger} \frac{1}{2} ( 1 +
\gamma_{4,\dirac}).\label{eq:chiral_proj}
\end{equation}
Using the definition of $U$ in eqn.~\ref{eq:sim_def}, we obtain the
solution to the non-relativistic quark propagator in the DR basis by solving
\begin{eqnarray}
M_\ch G' & = & S \frac{1}{\sqrt{2}}
\left(
\begin{array}{rr}
- i \sigma_2 & 0\\
-i \sigma_2 & 0
\end{array}
\right) \nonumber \\
& = & S \times \frac{1}{\sqrt{2}} \left( \begin{array}{rrrr}
0 & -1 & 0 & 0 \\
1 & 0 & 0 & 0 \\
0 & -1 & 0 & 0 \\
1 & 0 & 0 & 0
\end{array}
\right),
\end{eqnarray}
and then setting
\begin{equation}
G_\ch = G' U
\end{equation}
Clearly, the solution of eqn.~(\ref{eq:chiral_proj}) only involves the
upper two components of the source and the projection is explicit.
\end{document}















