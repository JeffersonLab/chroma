\documentclass[12pt]{article}
\usepackage{epsfig}
\newcommand{\pdotx}{\vec{p} \cdot \vec{x}}
\newcommand{\pdotz}{\vec{p} \cdot \vec{z}}
\newcommand{\qdoty}{\vec{q} \cdot \vec{y}}
\begin{document}
\hfill \today\\[1.0ex]
\noindent {\Large\bfseries Construction of 3pt.\ Meson Correlators}\\[1.0ex]

\section{Construction of Pion Three-point Function}
In this section, we describe how the pion form factor three-point
function in constructed from the propagators and extended propagators
computed within SZIN.  These notes should be merged with Randy's, in
particular regarding the normalisation of the propagators etc.

Following Randy's notation, we will consider the determination of the
form factor of the $\pi^+$, with operators
\begin{eqnarray}
\phi(x) & = & \bar{d}(x) \gamma_5 u(x)\\
\phi^{\dagger}(x) & = & - \bar{u}(x) \gamma_5 d(x)\\
V_{\mu}(x) & = & e_u \bar{u}(x) \gamma_{\mu} u(x) + e_d \bar{d}(x)
\gamma_{\mu} d(x).
\end{eqnarray}
The three-point correlator is then given by
\begin{equation}
\Gamma_{\pi^+ \mu \pi^+}(t_f, t; \vec{p}, \vec{q}) = \sum_{\vec{x},
\vec{y}} \langle 0 | \phi(\vec{x}, t_f) V_{\mu}(\vec{y}, t)
\phi^{\dagger}(\vec{0},0) | 0 \rangle e^{-i\pdotx}
e^{-i\vec{q}\cdot\vec{y}},
\end{equation}
where w.l.o.g.\ we place the initial pion at the origin.
\begin{figure}[h]
\begin{center}
\psfig{width=250pt,file=pion_three_point.eps}
\caption{Contribution of the $u$-quark part of the vector current to
  the pion form factor.  The dotted line shows the sequential-source
  propagator computed in the calculation.}
\end{center}
\end{figure}

Introducing the quark propagators for the $u$ and $d$ quarks as
follows
\begin{eqnarray}
U^{ij}_{\alpha\beta} (x,y) & = & \langle u^i_{\alpha}(x)
\bar{u}^j_{\beta}(y)\rangle\\
D^{ij}_{\alpha\beta} (x,y) & = & \langle d^i_{\alpha}(x)
\bar{d}^j_{\beta}(y)\rangle,
\end{eqnarray}
we will now separate the three point functions into the contributions
$\Gamma_{\pi^+ \mu \pi^+}^U$ and $\Gamma_{\pi^+ \mu \pi^+}^D$
arising from the insertion of the $u$ and $d$ quarks respectively,
\begin{eqnarray}
\Gamma_{\pi^+ \mu \pi^+}^U & = & e_u \sum_{\vec{x},\vec{y}} e^{-i
  \pdotx - i \vec{q}\cdot\vec{y}} {\rm Tr} \left\{
  \gamma_5 U(x,y) 
\gamma_{\mu} U(y,0)\gamma_5 D(0,x) \right\}\label{eq:u_quark_3pt}\\
\Gamma_{\pi^+ \mu \pi^+}^D & = & e_d \sum_{\vec{x},\vec{y}} e^{-i
  \pdotx - i \vec{q}\cdot\vec{y}} {\rm Tr} \left\{
  D(y,x) \gamma_5 U(x,0)\gamma_5 D(0,y)\gamma_{\mu} \right\}.
\label{eq:d_quark_3pt}
\end{eqnarray}
where the trace is over both the spinor and colour indices.
Note that there is no disconnected contribution to the
correlators~\cite{draper89}.

\section{Sequential-Source Propagators}
There are two ways of expressing the three-point correlators in terms
of sequential-source (or extended, or exponentiated) propagators.  In
the first method, we compute the sequential propagator with the
\textit{operator} as the sequential source; this method has the
advantage of allowing the matrix elements of a particular operator, at
a particular momentum, to be evaluated for \textit{any} external
states.  The second method employs the final hadron state as the
sequential source.  This method in general allows \textit{any} current
insertion to be evaluated for only \textit{particular} final and, in
the case of baryons, initial states, and is the method we are using in
this calculation.

\subsection{Lattice quark propagators.}
The general quark propagator
\begin{equation}
G^{ij}_{\alpha\beta}(x,y) = \langle q^i_{\alpha}(x)
\bar{q}^j_{\beta}(y)\rangle
\end{equation}
satisfies
\begin{equation}
M^{ik}_{\alpha\gamma}(x,z) G^{kj}_{\gamma\beta}(z,y) = \delta_{ij}
\delta_{\alpha \beta} \delta_{xy}\label{eq:dirac_eqn}
\end{equation}
where $M$ is the lattice Dirac operator.  The anti-quark propagator
is related through
\begin{equation}
G(y,x) = \gamma_5 G(x,y)^{\dagger} \gamma_5,\label{eq:anti_quark}
\end{equation}
which is a simple property of the lattice Dirac equation.
\textit{Note}: it is important to check for consistency in the
normalisation of the quark propagators.

\subsection{Sequential-source propagator for $u$ insertion.}
The SZIN code computes the following extended propagator
\begin{equation}
H^u (y,0; t_f, \vec{p}) = \sum_{\vec{x}} e^{i \pdotx} U(y,x) \gamma_5 D(x,0) \gamma_5\label{eq:u_insert}
\end{equation}
for the case of degenerate quark masses, i.e.\ equivalent $u$- and
$d$-quark propagators, and for a specified $\vec{p}$ and $t_f$.  Using
the Dirac equation, eqn.~\ref{eq:dirac_eqn}, we see that
\begin{equation}
M(z,y) H^u(y,0; t_f, \vec{p}) = \delta_{t_z, t_f} e^{i
  \pdotz} \gamma_5 D(z,0) \gamma_5.
  \label{eq:dirac_u_insert}
\end{equation}

Taking the complex conjugate of eqn.~\ref{eq:u_insert}, we have 
\begin{eqnarray}
H^u (y,0; t_f, \vec{p})^{\dagger} & = & \sum e^{-i\pdotx}\gamma_5
D^{\dagger}(x,0)\gamma_5 U^{\dagger}(y,x)\nonumber\\
& = & \sum e^{-i\pdotx} D(0,x) \gamma_5 U(x,y) \gamma_5,
\end{eqnarray}
using eqn.~\ref{eq:anti_quark}.  Finally, comparing with
eqn.~\ref{eq:u_quark_3pt}, and using the cyclic property of the trace,
we see that
\begin{equation}
\Gamma_{\pi^+ \mu \pi^+}^U = e_u \sum_{\vec{y}} e^{-i \qdoty} {\rm Tr}
\left\{ H^u(y,0; t_f, \vec{p})^{\dagger} \gamma_5 \gamma_{\mu} U(y,0)
\gamma_5\right\} \label{eq:u_quark_3pt_final}
\end{equation}

\subsection{Sequential-source propagator for $d$ insertion.}
This is computed the same way, with the r\^{o}les
of the $U$ and $D$ reversed, so that
\begin{equation}
H^d (y,0; t_f, \vec{p}) = \sum_{\vec{x}} e^{i \pdotx} D(y,x) \gamma_5
U(x,0) \gamma_5\label{eq:d_insert}
\end{equation}
whence
\begin{equation}
H^d (y,0; t_f, \vec{p})^{\dagger} = \sum e^{-i\pdotx}\gamma_5
U^{\dagger}(x,0)\gamma_5 D^{\dagger}(y,x)
\end{equation}

To see how we now construct the $d$-quark insertion in the propagator,
we now take the conjugate of eqn.~\ref{eq:d_quark_3pt}
\begin{eqnarray}
\Gamma_{\pi^+ \mu \pi^+}^D(t_f, t; \vec{p},\vec{q})^*  & =  & e_d
  \sum_{\vec{x},\vec{y}} e^{+i \pdotx + i \vec{q}\cdot\vec{y}} {\rm
  Tr} \left\{ \gamma_{\mu} D(0,y)^{\dagger} \gamma_5 U(x,0)^{\dagger}
  \gamma_5 D(y,x)^{\dagger}
  \right\}.\nonumber\\
& = & e_d \sum_{\vec{y}} e^{i\qdoty} {\rm Tr} \left\{ \gamma_{\mu}
  D(0,y)^{\dagger}  H^d(y,0; t_f, -\vec{p})^{\dagger}
  \right\}\nonumber\\
& = & - e_d \sum_{\vec{y}} e^{i\qdoty} {\rm Tr} \left\{ H^d(y,0;t_f,
  -\vec{p})^{\dagger} \gamma_5 \gamma_{\mu}
  D(y,0) \gamma_5 \right\}
\label{eq:d_quark_3pt_final}
\end{eqnarray}

Comparing eqns.~\ref{eq:u_quark_3pt_final} and
\ref{eq:d_quark_3pt_final}, we see that they are both of the same form
and involve only propagators computed in the SZIN code.  Finally, for
the case where $m_d = m_u \equiv m_q$, we see that both expressions
are essentially the same, and in particular
\begin{eqnarray}
\lefteqn{\sum_{\vec{y}} e^{i\qdoty} {\rm Tr} \left\{ H(y,0;t_f,
  \vec{p})^{\dagger} \gamma_5 \gamma_{\mu}
  Q(y,0) \gamma_5 \right\}\qquad  } \nonumber \\
   & = & \frac{1}{e_u} \Gamma^U_{\pi^+ \mu
  \pi^+}(t_f, t; \vec{p}, \vec{q})\nonumber \\ 
  & = & - \frac{1}{e_d} \Gamma^D_{\pi^+ \mu
  \pi^+}(t_f, t; - \vec{p}, - \vec{q})^*
\end{eqnarray}

\section{Calculation of $\rho \longrightarrow \pi$ transition form factors}

This form factor we can essentially get for free from the propagators
computed for the pion form factor.  UKQCD has a lot of experience in
doing this, in particular for the $B \longrightarrow \rho l \bar{\nu}$
transitions, so I have essentially just translated the UKQCD calculation to
the light-quark sector\cite{ukqcd95}

\subsection{Form Factor Definitions}
There is a single vector and three axial-vector transition form
factors:
\begin{eqnarray}
\langle \pi | (V_{\mu} - A_{\mu}) | \rho \rangle & = & T_{\mu\beta}
\epsilon_r^{\beta}\\
T_{\mu\beta} & = & \frac{2V(q^2)}{m_\pi + m_{\rho}}
\epsilon_{\mu\gamma\delta\beta} p^{\gamma}_{\pi} p^{\delta}_{\rho} + i
(m_{\pi} + m_{\rho}) A_1(q^2) g_{\mu\beta}+ \nonumber \\
& & i \frac{A_2(q^2)}{m_\pi + m_{\rho}}(p_\pi + \_\rho)_{\mu}
q_{\beta} + i \frac{A(q^2)}{q^2} 2 m_{\rho} q_{\mu} (p_\pi +
p_{\rho})_{\beta},
\end{eqnarray}
where $q = p^\pi - p^\rho$, $\epsilon$ is the $\rho$ polarisation
vector, and $r$ is the polarisation label.  There are several
competing definitions of the form factors, but we can postpone that
discussion until later.

\subsection{Two-point Functions}
The two-point function for the pion comes straight from the pion form
factor analysis that Randy has described:
\begin{eqnarray}
\Gamma_{\pi^+ \pi^+}(t_f; \vec{p}) & = & \sum_{\vec{x}} \langle
\phi_{\pi}(\vec{x}, t_f) \phi_{\pi}^{\dagger} (0) \rangle e^{- i
\vec{p}\cdot x}\\ & \longrightarrow & \frac{1}{2 E_{\pi}(\vec{p})}
|Z_P(|\vec{p}_{\pi}|)|^2 e^{- E_{\pi}(\vec{p}) t_f}.
\end{eqnarray}
In the case of the $\rho$ meson, we introducting the interpolating
operator
\begin{eqnarray}
\Omega_{\mu} & = & \bar{d} \gamma_\mu u \nonumber \\
\Omega_{\mu}^{\dagger} & = & \Delta_\mu \bar{u} \gamma_\mu d
\end{eqnarray}
where $\Delta_i = -1 (i = 1,2,3)$ and $\Delta_4 = +1$,
and determine the correlation function
\begin{equation}
\Gamma_{\rho^+ \rho^+, ij}(t_f, \vec{p}) = \sum_{\vec{x}} \langle
\Omega_i(\vec{x}, t_f) \Omega_j(0)^{\dagger} \rangle e^{-i\pdotx}.\label{eq:rho_2pt}
\end{equation}
{\cal N.B.} at non-zero momentum, the temporal component $\Omega_0$
also has an overlap with the vector meson; in principle we could use
it also, but I would be worried about the effect of smearing.

We define the overlap of the $\rho$ interpolating operator through
\begin{equation}
\langle 0 \mid \Omega_i(0) \mid \rho, r,\vec{p} \rangle = Z_V(|\vec{p}|)
\epsilon_i^r
\end{equation}
where $r$ is the polarization index.  I have always assumed, though
this should be checked, that this expression carries through
straightforwardly to the case of smeared interpolating operators, with
the complication that the overlap $Z_V$ depends on the spatial
momentum.  The polarization vectors satisfy
\begin{equation}
\sum_{r = 1,2} \epsilon_i^r(\vec{p}) \epsilon_j^r(\vec{p}) = \left(-
  g_{ij} + \frac{p_i p_j}{m_{\rho}^2}\right).
\end{equation}
Inserting into the correlator eqn.~\ref{eq:rho_2pt}, we have
\begin{eqnarray}
  \lefteqn{\Gamma_{\rho^+ \rho^+, ij}(t_f, \vec{p}) = } &  &\nonumber\\
&  & \sum_{N,r,\vec{k}}
\sum_{\vec{x}} \frac{1}{(2\pi)^3} \frac{1}{2 E_N(\vec{k})} \langle 0 |
\Omega_i(\vec{x}, t_f) | N,r,\vec{k} \rangle \langle N,r,\vec{k} |
\Omega_j(0)^{\dagger}|0 \rangle e^{-i\pdotx}.\nonumber\\
& = & \sum_{N,r,\vec{k}}
\sum_{\vec{x}} \frac{1}{(2\pi)^3} \frac{1}{2 E_N(\vec{k})} \langle 0 |
\Omega_i(0) | N,r,\vec{k} \rangle \langle N,r,\vec{k} |
\Omega_j(0)^{\dagger}|0 \rangle e^{-i\pdotx +i \vec{k}\cdot\vec{x} - i
E_N(\vec{k}) t_f}.\nonumber\\
& \longrightarrow & \sum_r \langle 0 |
\Omega_i | N,r,\vec{p} \rangle \langle N,r,\vec{p} |
\Omega_j^{\dagger}|0 \rangle e^{-i E_{\rho}(\vec{p}) t} + \mbox{higher
  terms}\nonumber \\
& = & \frac{1}{2 E_\rho(\vec{p})} \left( -\delta_{ij} + \frac{p_i
    p_j}{m_{\rho}^2} \right) |Z_V(|\vec{p}|)|^2 e^{-i E_{\rho}(\vec{p}
    )t_f}
\end{eqnarray}
Finally, summing over the spatial indices we have
\begin{equation}
\Gamma_{\rho^+ \rho^+}(t_f, \vec{p}) \equiv \sum_i \Gamma_{\rho^+
  \rho^+, ii}(t_f, \vec{p}) = -\frac{1}{2 E_{\rho}(\vec{p})} \left( 3 -
  \frac{\vec{p}^2}{m_{\rho}^2} \right) |Z_V(|\vec{p}|)|^2 e^{-i
  E_{\rho}(\vec{p}) t_f }
\end{equation}

\subsection{Three-point Functions}
We begin by considering
\begin{equation}
\Gamma^{(3)}_{\pi^+ \rho+^, \mu\nu}(t, t_f; \vec{p}, \vec{q}) = 
\sum_{\vec{x}, \vec{y}} \langle 0 \mid \phi(\vec{x}, t_f)
J_{\mu}(\vec{y}, t)
\Omega_{\nu}(0)^{\dagger} \mid 0 \rangle e^{-i\pdotx - i \qdoty}
\end{equation}
where $J_{\mu}$ is the inserted current.  Inserting complete sets of
states yields
\begin{eqnarray}
\lefteqn{\Gamma^{(3)}_{\pi^+ \rho+^, \mu\nu}(t, t_f; \vec{p}, \vec{q}) 
= \sum_{\vec{x}, \vec{y}} \sum_{M,\vec{k}} \sum_{N, \vec{l}, r}
\frac{1}{2 E_M(\vec{k})} \frac{1}{2 E_N(\vec{l})}
\frac{1}{(2\pi)^6}e^{-i\pdotx -i \qdoty}} & & \nonumber \\
& & \times
\langle 0 \mid \phi(\vec{x}, t_f) \mid M,\vec{k} \rangle \langle M,
\vec{k} \mid J_{\mu}(\vec{y},t) \mid N, \vec{l},r\rangle \langle N,
\vec{l} r \mid \Omega_{\nu}(0)^{\dagger} \mid 0 \rangle\nonumber\\
& = & \sum_M \sum_{N,r} \frac{1}{2 E_M(\vec{p})} \frac{1}{2
  E_N(\vec{p} + \vec{q})} e^{-i E_M(\vec{p}) t_f} e^{-i(E_N(\vec{p} +
  \vec{q}) - E_M(\vec{p})) t}\nonumber \\
& & \times \langle 0 \mid \phi \mid M, \vec{p}
\rangle \langle M,\vec{p} \mid J_\mu \mid N, \vec{p} + \vec{q}, r
\rangle \langle N, \vec{p} + \vec{q}, r \mid \Omega_{\nu} \mid 0
\rangle \nonumber \\
& = & \sum_M \sum_{N,r} \frac{1}{2 E_M(\vec{p})}\frac{1}{2 E_N(\vec{p}
  + \vec{q})} e^{-iE_M(\vec{p}) (t_f - t) -i E_N(\vec{p} +
  \vec{q})t}\nonumber \\
& & \times Z_P(|\vec{p}|) Z_V(|\vec{p} + \vec{q}|)^* \epsilon^r_{\nu}
T_{\mu\beta} \epsilon^{r,\beta} \nonumber \\
& = & \sum_{M,N} \frac{1}{4 E_M(\vec{p}) E_N(\vec{p}+\vec{q})}
e^{-iE_M(\vec{p}) (t_f - t) -i E_N(\vec{p} + \vec{q})t} \nonumber \\
& & \times
Z_P(|\vec{p}|) Z_V(|\vec{p}+\vec{q}|) T_{\mu\beta} \left( -
\delta_\nu^\beta + \frac{p_{N\nu} p_N^\beta}{m_N^2} \right)
%
\end{eqnarray}

\subsection{Transition form factors and sequential source propagators}
We can use the same sequential propagators for the transition matrix
element that we used for the pion form factor.  We write the correlors
in terms of the $u$- and $d$-quark contributions as
\begin{equation}
\Gamma_{\pi^+ \rho^+, \mu,\nu} = \Gamma_{\pi^+ \rho^+, \mu,\nu}^U +
\Gamma_{\pi^+ \rho^+, \mu,\nu}^D.
\end{equation}

\subsubsection{Sequential sources for $\Gamma^U$}
We have
\begin{eqnarray}
\Gamma^U_{\pi^+ \rho+^, \mu\nu}(t, t_f; \vec{p}, \vec{q}) & = &
e_u \Delta_{\nu} \sum_{\vec{x}, \vec{y}} \langle \bar{d}(x) \gamma_5
u(x) \bar{u}(y) \Gamma_\mu u(y) \bar{u}(0) \gamma_\nu d(0) \rangle
e^{-i\pdotx - i \qdoty}\nonumber \\
& = & - e_u \Delta_\nu \sum_{\vec{x},\vec{y}} {\rm Tr} \left\{
\gamma_5 U(x,y) \Gamma_\mu U(y,0) \gamma_{\nu} D(0,x) \right\}.
\end{eqnarray}
Using the definiting of the sequential source propagator in
eqn.~\ref{eq:u_insert}, we have
\begin{equation}
\Gamma^U_{\pi^+ \rho^+, \mu\nu} = -e_u \Delta_{\nu} \sum_{\vec{y}}
e^{-i\qdoty} {\rm Tr} \left\{ H^u(y,0; t_f, \vec{p})^{\dagger}
\gamma_5 \Gamma_\mu U(y,0) \gamma_{\nu} \right\}
\end{equation}

\subsubsection{Sequential sources for $\Gamma^D$}
We have
\begin{eqnarray}
\Gamma^D_{\pi^+ \rho+^, \mu\nu}(t, t_f; \vec{p}, \vec{q}) & = &
e_d \Delta_{\nu} \sum_{\vec{x}, \vec{y}} \langle \bar{d}(x) \gamma_5
u(x) \bar{d}(y) \Gamma_\mu d(y) \bar{u}(0) \gamma_nu d(0) \rangle
e^{-i\pdotx - i \qdoty}\nonumber \nonumber \\
& = & - e_d \Delta_\nu \sum_{\vec{x},\vec{y}} {\rm Tr} \left\{ D(y,x)
\gamma_5 U(x,0) \gamma_\nu D(0,y) \Gamma_\mu \right\}.
\end{eqnarray}
Taking the complex conjugate we have
\begin{eqnarray}
\Gamma^D_{\pi^+ \rho+^, \mu\nu}(t, t_f; \vec{p}, \vec{q})^* & = &
- e_d \Delta_\nu \sum_{\vec{x},\vec{y}} e^{i\pdotx + i \qdoty}{\rm Tr} \left\{
\Gamma_\mu^{\dagger} D(0,y)^{\dagger} \gamma_\nu^{\dagger}
U(x,0)^{\dagger} \gamma_5^{\dagger} D(y,x)^{\dagger} \right\}\nonumber
\\
& = & -e_d \Delta_{\nu} \sum_{\vec{y}} e^{i\qdoty} {\rm Tr} \left\{
\Gamma_{\mu}^{\dagger}  D(0,y)^{\dagger} \gamma_\nu \gamma_5 H^d(y,0;
t_f, -\vec{p}) \right\}\nonumber \\
& = & e_d \Delta_{\nu} \sum_{\vec{y}} e^{i \qdoty} {\rm Tr} \left\{
\Gamma_{\mu}^{\dagger} \gamma_5 D(y,0) \gamma_{\nu} H^d(y,0; t_f,
-\vec{p}) \right\},
\end{eqnarray}
where we have used the extended $d$-quark propagagator defined in
eqn.~\ref{eq:d_insert}.

\begin{thebibliography}{1}
\bibitem{draper89} T.~Draper \textit{et al}, Nucl.\ Phys.\ B318, 319
  (1989).
\bibitem{ukqcd95} K.C.~Bowler \textit{et al}, Phys.\ Rev.\ D51, 4905
  (1995).
\end{thebibliography}
\end{document}