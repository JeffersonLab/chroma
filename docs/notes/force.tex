\documentclass[12pt]{article}
\usepackage{latexsym}
\usepackage{times}
\usepackage{fullpage}
\title{HMC Force term for two flavours of Wilson Fermions}
\author{B\'alint Jo\'o}



\begin{document}
\newcommand{\Tr}{\mbox{Tr}\ } 
\newcommand{\Trs}{\mbox{Tr}_{s} \ }
\newcommand{\Trc}{\mbox{Tr}_{c} \ }
\newcommand{\Mdag}{M^{\dagger}}
\newcommand{\Udag}{U^{\dagger}}
\newcommand{\Ddag}{D^{\dagger}}
\newcommand{\Udot}{\dot{U}}
\newcommand{\UdagDot}{\dot{U}^{\dagger}}
\newcommand{\Mdot}{\dot{M}}
\newcommand{\MdagDot}{\dot{M}^{\dagger}}
\newcommand{\Ddot}{\dot{D}}
\newcommand{\DdagDot}{\dot{D}^{\dagger}}
\newcommand{\pidot}{\dot{\pi}}
\newcommand{\Hdot}{\dot{H}}
\newcommand{\Sdot}{\dot{S}}
\newcommand{\phidag}{\phi^{\dagger}}
\newcommand{\Xdag}{X^{\dagger}}
\newcommand{\Ydag}{Y^{\dagger}}
\newcommand{\psidag}{\psi^{\dagger}}
\newcommand{\sigmadag}{\sigma^{\dagger}}
\newcommand{\rhodag}{\rho^{\dagger}}
\newcommand{\onepgm}{ (1 + \gamma_{\mu}) }
\newcommand{\onemgm}{ (1 - \gamma_{\mu}) }
\maketitle
\abstract{We derive the HMC force term for two flavours of Wilson fermions
for the unpreconditioned and the preconditioned case, following the thesis
appendix of \cite{Zbysh}}

\section{Preliminaries, and Notation}
In this derivation We follow Zbyshek's thesis. The plan is to write down 
the two flavour fermionic Hamiltonian, find its derivative with respect
to molecular dynamics time and set it to zero. This will give us the force
term. We ignore the gauge piece here, but it could be included too. 

We will use the usual notation $U_{\mu,x}$ for a gauge field on site
$x$ pointing in the direction $\mu$. Likewise, the momenta will be denoted
by $\pi_{\mu,x}$. The Wilson Fermion matrix will be denoted $M$, and the
hopping term will be denoted $D$ to stop me having to figure out the 
LaTeX-ery to put the slash through it. Time derivatives with respect to 
Molecular Dynamics time $\tau$ will be denoted
(following Zbysh's thesis) in ``dot'' notation, ie:
\begin{equation}
\frac{d U_{\mu,x}}{d \tau} = \Udot_{\mu,x}
\end{equation}

Checkerboarding notation, where needed, will be denoted by subscripts
$0$ and $1$ referring to the checkerboard index in question.

\subsection{Hamiltonian}

We consider the Hamiltonian
\begin{equation}
H = \frac{1}{2}\sum_{\mu,x} \Tr \pi_{\mu,x}^2 + S(U)
\end{equation}
with  $S(U)$ being the action.

The equations of motion of $U_{\mu x}$ with respect to a fictitious 
molecular dynamics time take the form:
\begin{eqnarray}
  \Udot_{\mu,x} &=& \pi_{\mu,x} U_{\mu,x}  \label{e:Udot} \\
  \UdagDot_{\mu,x} &=& - \Udag_{\mu,x} \pi_{\mu,x} \label{e:UDagDot} \\
\end{eqnarray}
where eq (\ref{e:UDagDot}) comes from the fact that the $\pi$ are 
antihermitian. (Note: Zbysh's thesis used momenta $P_{\mu,x}$ defined
to be hermitian and he had a factor of $i$ floating about. The 
correspondance between his $P$ and our $\pi$ is $\pi=iP$.)

We can write the time derivative of the Hamiltonian as
\begin{equation}
\dot{H} = \sum_{x,\mu} \Tr \pi_{\mu,x} \pidot_{\mu,x} + \Sdot(U)
\end{equation}
and what we will show is that $\Sdot$ can be written
in the form
\begin{equation}
\Sdot = \sum_{\mu, x} \Tr \pi_{\mu,x} F_{\mu,x}
\end{equation}
for some $F_{\mu x}$. Then 
\begin{equation}
\Hdot = \sum_{x,\mu} \Tr \pi_{\mu,x} \left( \pidot_{\mu,x} + F_{\mu,x} \right) \ .
\end{equation}
Now since 
\begin{equation}
\Tr \pi_{\mu,x} = 0
\end{equation}
we can ensure that $\Hdot$ is zero if 
\begin{equation}
\pidot_{\mu,x} + F_{\mu,x} \propto 1 \ .
\end{equation}
We can ensure that the $\pidot$ are traceless if we choose the
constant of proportionality to be $\Tr F_{\mu x}$ so that
\begin{equation}
\pidot = -F_{\mu,x} + \Tr F_{\mu,x} = - \mathcal{F}_{\mu,x} \label{e:pidot}
\end{equation}
where
\begin{equation}
\mathcal{F}_{\mu,x} = F_{\mu,x} - \Tr F_{\mu,x} \label{e:mathF}
\end{equation}
is the force term.

The rest of this document is concerned with finding $\Sdot$ and $F$
for Wilson fermions.

\subsection{Two Flavour Fermion Actions}
For both the preconditioned and the unpreconditioned case we can write
the two flavour fermion action as:
\begin{equation}
S = \chi^{\dagger} (\Mdag M )^{-1} \chi
\end{equation}
with the difference between the two actions being the form of the 
matrix $M$ and on what part of the lattice the pseudofermion fields
are defined.

We can therefore write $\Sdot$ as
\begin{eqnarray}
\Sdot & = & \frac{d}{d \tau} \left( \chi^{\dagger} (\Mdag M)^{-1} \chi \right) \\
      & = & - \chi^{\dagger} \left( (\Mdag M)^{-1} \frac{d}{d \tau} \left[ \Mdag M \right] (\Mdag M)^{-1} \chi \right) \\
      & = & - \Xdag \left( \MdagDot M + \Mdag \Mdot \right) X \\
      & = & - \Xdag \MdagDot Y - \Ydag \Mdot X \label{e:SdotXY}
\end{eqnarray}
where we have defined vectors:
\begin{eqnarray}
  X &=& (\Mdag M)^{-1} \chi \\
  Y &=& M X = (\Mdag)^{-1} \chi
\end{eqnarray}

\section{The Unpreconditioned Case}
We define the unpreconditioned Wilson fermion matrix as:
\begin{equation}
M = (N_d + m) - \frac{1}{2} \mathcal{D}, \qquad \Mdag = (N_d + m) - \frac{1}{2} \mathcal{D}^{\dagger} \ ,
\end{equation}
with 
\begin{equation}
\mathcal{D} = \left[ \begin{array}{cc} 0 & D \\ D & 0 \end{array} \right], \qquad \mathcal{D}^{\dagger} = \left[ \begin{array}{cc} 0 & D^{\dagger} \\ D^{\dagger} & 0 \end{array} \right]
\end{equation}
and so
\begin{equation}
\Mdot = -\frac{1}{2} \dot{\mathcal{D}}, \qquad \MdagDot = -\frac{1}{2} \dot{\mathcal{D}}^{\dagger}
\end{equation}
From (\ref{e:SdotXY}) we have:
\begin{eqnarray} 
\Sdot &=&  - \Xdag \MdagDot Y - \Ydag \Mdot X \\
      &=& \frac{1}{2} \left( \Xdag \dot{\mathcal{D}}^{\dagger} Y + \Ydag \dot{\mathcal{D}} X \right) 
\end{eqnarray}

I can now write 
\begin{equation}
X = \left[ \begin{array}{c} X_0 \\ X_1 \end{array} \right],\ Y = \left[ \begin{array}{c} Y_0 \\ Y_1 \end{array} \right],\ \Xdag = [ \Xdag_0, \Xdag_1 ], \ \Ydag = [ \Ydag_0, \Ydag_1 ]
\end{equation}
and write 
\begin{equation}
\Sdot = \frac{1}{2} \left( \Xdag_0 \DdagDot Y_1 + \Xdag_1 \DdagDot Y_0 + \Ydag_0 \Ddot X_1 + \Ydag_1 \Ddot X_0 \right)
\end{equation}

Now
\begin{eqnarray}
\Ddot_{x,y}&=&\sum_{\mu} \onemgm \Udot_{\mu,x} \delta_{y,x+\mu} + \onepgm \UdagDot_{\mu,x-\mu} \delta_{y,x-\mu} \label{e:Ddot}\\
\DdagDot_{x,y} &=& \sum_{\mu} \onepgm \Udot_{\mu,x} \delta_{y,x+\mu} +  \onemgm \UdagDot_{\mu,x-\mu} \delta_{y,x-\mu} \label{e:DdagDot}
\end{eqnarray}

Let us define 
\begin{eqnarray} 
\psi &=& X_0 \\
\phi &=& Y_0 \\
\rho &=& X_1 \\
\sigma   &=& Y_1 \\
\end{eqnarray}
so that we do not need to worry about checkerboard subscripts.
Then 
\begin{equation}
\Sdot = \frac{1}{2} \left( \psidag \DdagDot \sigma + \rhodag \DdagDot \phi + \phidag \Ddot \rho + \sigmadag \Ddot \psi \right) \label{e:Ssigmasig}
\end{equation}

Let us now consider each of these terms in turn:
\begin{eqnarray}
\psidag \DdagDot \sigma &=& \sum_{x,\mu}  \psidag_{x} \onemgm \UdagDot_{\mu, x-\mu} \sigma_{x-\mu} +\psidag_{x} \onepgm \Udot_{\mu,x} \sigma_{x+\mu} \\
%
\rhodag \DdagDot \phi &=& \sum_{x,\mu}  \rhodag_{x} \onemgm \UdagDot_{\mu, x-\mu} \phi_{x-\mu} +\rhodag_{x} \onepgm \Udot_{\mu,x} \phi_{x+\mu}  \\
%
\phidag \Ddot \rho &=& \sum_{x,\mu} \phidag_x \onemgm \Udot_{\mu,x} \rho_{x+\mu} + \phidag_{x} \onepgm \UdagDot_{\mu, x-\mu} \rho_{x-\mu} \\
%
\sigmadag \Ddot \psi &=&  \sum_{x,\mu} \sigmadag_x \onemgm \Udot_{\mu,x} \psi_{x+\mu} + \sigmadag_{x} \onepgm \UdagDot_{\mu, x-\mu} \psi_{x-\mu} \\
\end{eqnarray}

We now use $\Udot_{\mu,x} = \pi_{\mu,x} U_{\mu, x}$ and $\UdagDot_{\mu,x}= -\Udag_{\mu,x} \pi_{\mu,x}$.
\begin{eqnarray}
\psidag \DdagDot \sigma &=& \sum_{x,\mu} \psidag_{x} \onepgm \pi_{\mu,x} U_{\mu,x}  \sigma_{x+\mu} - \psidag_{x} \onemgm \Udag_{\mu, x-\mu} \pi_{\mu, x-\mu} \sigma_{x-\mu}  \\
%
\rhodag \DdagDot \phi &=& \sum_{x,\mu} \rhodag_{x} \onepgm \pi_{\mu,x} U_{\mu,x}  \phi_{x+\mu} - \rhodag_{x} \onemgm \Udag_{\mu, x-\mu} \pi_{\mu, x-\mu} \phi_{x-\mu}  \\
%
\phidag \Ddot \rho &=& \sum_{x,\mu} - \phidag_{x} \onepgm \Udag_{\mu, x-\mu} \pi_{\mu, x-\mu} \rho_{x-\mu} + \phidag_x \onemgm \pi_{\mu,x} U_{\mu,x} \rho_{x+\mu}  \\
%
\sigmadag \Ddot \psi &=& \sum_{x,\mu} - \sigmadag_{x} \onepgm \Udag_{\mu, x-\mu} \pi_{\mu, x-\mu} \psi_{x-\mu} + \sigmadag_x \onemgm \pi_{\mu,x} U_{\mu,x} \psi_{x+\mu}  
\end{eqnarray}

We now use the fact that the sum over $x$ makes the expressions translation invariant, and use the $x \rightarrow x+\mu$ in the terms involving $x-\mu$, to get

\begin{eqnarray}
\psidag \DdagDot \sigma &=& \sum_{x,\mu} \psidag_{x} \onepgm \pi_{\mu,x} U_{\mu,x}  \sigma_{x+\mu} - \psidag_{x+\mu} \onemgm \Udag_{\mu, x} \pi_{\mu, x} \sigma_{x} \\
%
\rhodag \DdagDot \phi &=& \sum_{x,\mu} \rhodag_{x} \onepgm \pi_{\mu,x} U_{\mu,x}  \phi_{x+\mu} - \rhodag_{x+\mu} \onemgm \Udag_{\mu, x} \pi_{\mu, x} \phi_{x}  \\
%
\phidag \Ddot \rho &=& \sum_{x,\mu} - \phidag_{x+\mu} \onepgm \Udag_{\mu, x} \pi_{\mu, x} \rho_{x} + \phidag_x \onemgm \pi_{\mu,x} U_{\mu,x} \rho_{x+\mu}  \\
%
\sigmadag \Ddot \psi &=& \sum_{x,\mu} - \sigmadag_{x+\mu} \onepgm \Udag_{\mu, x} \pi_{\mu, x} \psi_{x} + \sigmadag_x \onemgm \pi_{\mu,x} U_{\mu,x} \psi_{x+\mu}  
\end{eqnarray}
Using the identity $Q^{\dagger}R = \Tr R Q^{\dagger}$ and using the cyclic 
properties of traces, the above terms can be reordered as:
\begin{eqnarray}
\psidag \DdagDot \sigma &=& \sum_{x,\mu} \Tr \pi_{\mu,x} U_{\mu,x}  \sigma_{x+\mu} \psidag_{x}  \onepgm  - \Tr  \Udag_{\mu, x} \pi_{\mu, x} \sigma_{x} \psidag_{x+\mu} \onemgm \\
%
\rhodag \DdagDot \phi &=& \sum_{x,\mu} \Tr  \pi_{\mu,x} U_{\mu,x}  \phi_{x+\mu} \rhodag_{x}  \onepgm - \Tr \Udag_{\mu, x} \pi_{\mu, x} \phi_{x} \rhodag_{x+\mu} \onemgm  \\
%
\phidag \Ddot \rho &=& \sum_{x,\mu} - \Tr \Udag_{\mu, x}  \pi_{\mu, x} \rho_{x} \phidag_{x+\mu} \onepgm  + \Tr \pi_{\mu,x} U_{\mu,x} \rho_{x+\mu} \phidag_x \onemgm  \\
%
\sigmadag \Ddot \psi &=& \sum_{x,\mu} - \Tr \Udag_{\mu, x} \pi_{\mu, x} \psi_{x} \sigmadag_{x+\mu} \onepgm  + \Tr  \pi_{\mu,x} U_{\mu,x} \psi_{x+\mu} \sigmadag_x \onemgm 
\end{eqnarray}
and so

\begin{eqnarray} 
\Sdot &=& \frac{1}{2} \left\{ \sum_{x,\mu} \Tr \pi_{\mu,x} U_{\mu,x}  \sigma_{x+\mu} \psidag_{x}  \onepgm  - \Tr  \Udag_{\mu, x} \pi_{\mu, x} \sigma_{x} \psidag_{x+\mu} \onemgm \right. \nonumber \\
&+&  \sum_{x,\mu} \Tr  \pi_{\mu,x} U_{\mu,x}  \phi_{x+\mu} \rhodag_{x}  \onepgm - \Tr \Udag_{\mu, x} \pi_{\mu, x} \phi_{x} \rhodag_{x+\mu} \onemgm  \nonumber \\
&+&  \sum_{x,\mu} - \Tr \Udag_{\mu, x}  \pi_{\mu, x} \rho_{x} \phidag_{x+\mu} \onepgm  + \Tr \pi_{\mu,x} U_{\mu,x} \rho_{x+\mu} \phidag_x \onemgm  \nonumber \\
&+& \left. \sum_{x,\mu} - \Tr \Udag_{\mu, x} \pi_{\mu, x} \psi_{x} \sigmadag_{x+\mu} \onepgm  + \Tr  \pi_{\mu,x} U_{\mu,x} \psi_{x+\mu} \sigmadag_x \onemgm \right\} \\
%
  &=& \frac{1}{2} \left\{ \sum_{x,\mu} \Tr \pi_{\mu,x} U_{\mu,x}  \sigma_{x+\mu} \psidag_{x}  \onepgm   + \Tr \pi_{\mu,x} U_{\mu,x} \rho_{x+\mu} \phidag_x \onemgm   \nonumber \right. \\
&+&  \sum_{x,\mu} \Tr  \pi_{\mu,x} U_{\mu,x}  \phi_{x+\mu} \rhodag_{x}  \onepgm   + \Tr  \pi_{\mu,x} U_{\mu,x} \psi_{x+\mu} \sigmadag_x \onemgm \nonumber \\
&-& \left(\sum_{x,\mu} \Tr \Udag_{\mu, x} \pi_{\mu, x} \psi_{x} \sigmadag_{x+\mu} \onepgm  + \Tr \Udag_{\mu, x} \pi_{\mu, x} \phi_{x} \rhodag_{x+\mu} \onemgm \right) \nonumber \\
&-& \left. \left( \sum_{x,\mu} \Tr \Udag_{\mu, x}  \pi_{\mu, x} \rho_{x} \phidag_{x+\mu} \onepgm + \Tr  \Udag_{\mu, x} \pi_{\mu, x} \sigma_{x} \psidag_{x+\mu} \onemgm \right) \right\} \\
%
&=&  \frac{1}{2} \left\{ \sum_{x,\mu} \Tr \pi_{\mu,x} U_{\mu,x} \left( \phi_{x+\mu} \rhodag_{x} + \sigma_{x+\mu} \psidag_{x} \right)  \onepgm   \nonumber \right. \\
& + & \left. \Tr \pi_{\mu,x} U_{\mu,x} \left( \psi_{x+\mu} \sigmadag_x + \rho_{x+\mu} \phidag_x \right) \onemgm  +  \mbox{ Hermitian conjugate } \right\} 
\end{eqnarray}
Which is of the desired form
\begin{equation}
\Sdot = \sum_{x,\mu} \Tr \pi_{\mu,x} F_{\mu,x}
\end{equation} 
with
\begin{equation}
F_{x,\mu} =\frac{1}{2} \left\{ U_{\mu,x} \left[ \left( \phi_{x+\mu} \rhodag_{x} + \sigma_{x+\mu} \psidag_{x} \right)  \onepgm +  \left( \psi_{x+\mu} \sigmadag_x + \rho_{x+\mu} \phidag_x \right) \onemgm \right] + \mbox{h.c} \right\}
\end{equation}

We can judiciously ignore the hermitian conjugate terms as the end result
of the momentum update will project them out anyway using $\tt taproj$.

\subsection{Checkerboarding}
One thing we have neglected so far is the issue of checkerboarding. Let us 
consider this now. 

Where $U_{\mu,x}$ is multiplied into $F_{\mu,x}$ it must have the same checkerboard index as the unshifted field, whereas the shifted field will come from the
other checkerboard. 
So 
\begin{itemize}
\item
for the $\phi_{x+\mu} \rhodag_{x}$ term the $U_{\mu,x}$ must have the same checkerboard index as the $\rhodag$.
\item 
for the $\sigma_{x+\mu} \psidag_{x}$ term the $U_{\mu,x}$ must have the same checkerboard index as the $\psidag$.
\item
for the $\psi_{x+\mu} \sigmadag_{x}$ term the $U_{\mu,x}$ must have the same checkerboard index as the $\sigmadag_{x}$
\item
for the $\rho_{x+\mu} \psidag_{x}$ term the $U_{\mu,x}$ must have the same checkerboard index as the $\psidag_{x}$
\end{itemize}

Now since the $\psi$ and $\phi$ terms are defined on one checkerboard and 
the $\rho$ and $\sigma$ terms are defined on the opposite one, we can split
the force term into two terms: 
\begin{equation}
F_{\mu,x} = F^{0}_{\mu,x} + F^{1}_{\mu,x}
\end{equation}

In our particular case $\psi$ and $\phi$ are defined on checkerboard index
0, and $\rho$ and $\sigma$ are defined on checkerboard index 1. So
\begin{eqnarray}
F^{0}_{\mu,x} &=& \frac{1}{2} U^{0}_{\mu,x} \left[ \sigma_{x+\mu} \psidag_{x} \left( 1 + \gamma_{\mu} \right) + \rho_{x+\mu} \phidag_{x} \left( 1 - \gamma_{\mu} \right) \right] \\
F^{1}_{\mu,x} &=& \frac{1}{2} U^{1}_{\mu,x} \left[ \phi_{x+\mu} \rhodag_{x} \left( 1 + \gamma_{\mu} \right) + \psi_{x + \mu} \sigmadag_{x} \left( 1 - \gamma_{\mu} \right) \right]
\end{eqnarray}

If we now put back our definitions of $\psi, \phi, \sigma$ and $\rho$, 
we can see that:
\begin{eqnarray}
 F^{0}_{\mu,x} &=& \frac{1}{2} U^{0}_{\mu,x} \left[ Y^{1}_{x+\mu} X^{0 \dagger}_{x} \left( 1 + \gamma_{\mu} \right) + X^1_{x+\mu} Y^{ 0\dagger}_{x} \left( 1 - \gamma_{\mu} \right) \right] \\
F^{1}_{\mu,x} &=& \frac{1}{2} U^{1}_{\mu,x} \left[ Y^0_{x+\mu} X^{1 \dagger}_{x} \left( 1 + \gamma_{\mu} \right) + X^{0}_{x + \mu} Y^{1 \dagger }_{x} \left( 1 - \gamma_{\mu} \right) \right]
\end{eqnarray}

and we realise (before committing suicide) that we can do the whole shebang
by not checkerboarding at all and just computing:
\begin{eqnarray}
F_{\mu,x} &=& \frac{1}{2} U_{\mu, x} \left( Y_{x+\mu}\Xdag_{x} \onepgm + X_{x+\mu} \Ydag_{x} \onemgm \right) 
\end{eqnarray} 

\section{The Preconditioned Case}
The $M$ for the preconditioned case is
\begin{equation} 
M = (N_d + m) - \frac{1}{4(N_d + m)} D D 
\end{equation}
and
\begin{equation}
\Mdag = (N_d + m) - \frac{1}{4(N_d + m)} \Ddag \Ddag
\end{equation}
 acting only on one subset of the lattice with checkerboard index $p$. So
\begin{equation}
\Mdot = -\frac{1}{4(N_d + m) }\left( \Ddot D + D \Ddot \right), \MdagDot = -\frac{1}{4(N_d + m)} \left( \DdagDot \Ddag + \Ddag \DdagDot \right)
\end{equation}
and
\begin{eqnarray}
\Sdot &=& \frac{1}{4(N_d + m)} \Xdag \left( \DdagDot \Ddag + \Ddag \DdagDot \right) Y + \frac{1}{4(N_d + m)} \Ydag \left(  \Ddot D + D \Ddot \right) X \\
      &=& \frac{1}{4(N_d + m)} \left( \Xdag \DdagDot \Ddag Y + \Xdag \Ddag \DdagDot Y + \Ydag \Ddot D X + \Ydag D \Ddot X \right)
\end{eqnarray}
where $X$ and $Y$ are now only defined on the relevant checkerboarded sites of the lattice (with index $p$)

As before we define
\begin{eqnarray} 
  \psi^{p} &=& X^p \\
  \phi^{p} &=& Y^p \\
  \rho^{1-p} &=& D X^p \\
  \sigma^{1-p}   &=& \Ddag Y^p
\end{eqnarray}
so that (neglecting the checkerboard indices for the moment)
\begin{equation}
\Sdot = \frac{1}{4(N_d + m)} \left( \psidag \DdagDot \sigma + \rhodag \DdagDot \phi + \phidag \Ddot \rho + \sigmadag \Ddot \psi \right) 
\end{equation}

The terms are of the same form as the unpreconditioned case, and apart 
from the factor of $\frac{1}{4(N_d + m)}$ and the different definitions of the $\psi, \phi, \rho$ and $\sigma$, the manipilations of the terms are the same as the 
unpreconditioned case. So we can write down immediately that:

\begin{eqnarray}
F_{x,\mu} &=& \frac{1}{4(N_d + m)} \nonumber \\
&\times& \left\{ U_{\mu,x} \left[ \left( \phi_{x+\mu} \rhodag_{x} + \sigma_{x+\mu} \psidag_{x} \right)  \onepgm +  \left( \psi_{x+\mu} \sigmadag_x + \rho_{x+\mu} \phidag_x \right) \onemgm \right] + \mbox{h.c} \right\} \nonumber \\
& &
\end{eqnarray}

\subsection{Checkerboarding}
The $D$ term in the definitions of $\sigma$ and $\rho$ imply that these
fields are defined on the opposite checkerboard from the $X$ and $Y$ fields.
Once again (ignoring the hermitian conjugate terms) we can split the force 
into two terms which we can index with the checkerboard index of the 
unshifted fields.
\begin{eqnarray}
F^{p}_{\mu,x} &=& \frac{1}{4(N_d + m)} U^{p}_{\mu,x} \left[ \sigma_{x+\mu} \psidag_{x} \left( 1 + \gamma_{\mu} \right) + \rho_{x+\mu} \phidag_{x} \left( 1 - \gamma_{\mu} \right) \right] \\
F^{1-p}_{\mu,x} &=& \frac{1}{4(N_d + m} U^{1-p}_{\mu,x} \left[ \phi_{x+\mu} \rhodag_{x} \left( 1 + \gamma_{\mu} \right) + \psi_{x + \mu} \sigmadag_{x} \left( 1 - \gamma_{\mu} \right) \right]
\end{eqnarray}
We note that in both Chroma and SZIN for the preconditioned fermion matrix
the $X$ and $Y$ fields are defined on $p=1$.

\section{SZIN}
In SZIN convention where the preconditioned matrix is
\begin{equation}
M = 1 - \kappa^2 D D
\end{equation}
defined on checkerboard index 1. ($p=1$).
Only the prefactors change and so we should find that:
\begin{eqnarray}
F_{x,\mu} &=& \kappa^2 \left\{ U_{\mu,x} \left[ \left( \phi_{x+\mu} \rhodag_{x} + \sigma_{x+\mu} \psidag_{x} \right)  \onepgm +  \left( \psi_{x+\mu} \sigmadag_x + \rho_{x+\mu} \phidag_x \right) \onemgm \right] + \mbox{h.c} \right\} \nonumber \\
& & \label{e:SZINForceTerm}
\end{eqnarray}
and since we have fixed the checkerboard index of $X$ and $Y$ to be $1$ we can
write down the two checkerboarded force components with $F^{0}$ corresponding 
to the unshifted $\sigma$ and $\rho$ fields, and $F^{1}$ corresponding to the 
unshifted $\psi$ and $\phi$ fields:
\begin{eqnarray}
F^{0}_{\mu,x} &=& \kappa^2 U^{0}_{\mu,x} \left[ \psi_{x + \mu} \sigmadag_{x} \left( 1 - \gamma_{\mu} \right) + \phi_{x+\mu} \rhodag_{x} \left( 1 + \gamma_{\mu} \right) \right] \label{e:SZINcb0} \\
F^{1}_{\mu,x} &=& \kappa^2 U^{1}_{\mu,x} \left[  \rho_{x+\mu} \phidag_{x} \left( 1 - \gamma_{\mu} \right) + \sigma_{x+\mu} \psidag_{x} \left( 1 + \gamma_{\mu} \right) \right] \label{e:SZINcb1} 
\end{eqnarray}

The pseudocode for the SZIN force computation is described in figure
\ref{f:SZINPseudo}. We can see that the terms constructed are the ones
in (\ref{e:SZINcb0}) and (\ref{e:SZINcb1}) except the $(1 \pm
\gamma_{\mu})$ terms are in the front rather than the rear -- this is
OK in the final traces. The $\kappa^2$ factor is included by
incorporating them into the $\phi$ and $\sigma$ terms. SZIN does not
calculate the hermitian conjugate term, and the SZIN force picks up an
overall minus sign relative to $F$, which is OK since in {\tt leapp.m}
the Force term is subtracted. The lack of the hermitian conjugate term
is probably OK since the momenta will be projected using taproj (which
will also remove the traceful parts)

\begin{figure}[ht]
\label{f:SZINPseudo}
\begin{center}
\leavevmode
\hbox{ 
\begin{tabular}{ll} 
$ \psi = X_1        $       &  // Input                   \\
$ \phi = M \psi   $       &  // $\phi_1 = Y_1 $    \\
$ \rho = D_{01} \psi_1   $       &  // $\rho_0 = DX_1 $  \\ 
$ \phi *= -\kappa^2 $     &  // $\phi_1 = -\kappa^2 Y_1$   \\
$ \sigma = D_{01}^{\dagger} \phi_{1}$ & // $\sigma_0 = -\kappa^2 D^{\dagger}Y_1$ \\
for($\mu = 0$, $\mu < N_d$; $\mu++$) \{ &                  \\
\quad Checkerboard 0                   &                  \\
\quad  $ftmp\_2 \ = -(1-\gamma_{\mu}) \psi$    &                    \\
\quad  $utmp\_1 \ = - ftmp\_2_{x+\mu} \sigma^{\dagger}$ &  // $-\kappa^2(1-\gamma_{\mu})\psi_{x+\mu} \sigmadag_x$       \\
\quad  $ftmp\_2 \ = (1 + \gamma_{\mu}) \phi$  &                    \\
\quad  $utmp\_1 \ += ftmp\_2_{x+\mu} \rho^{\dagger}$ & // $-\kappa^2 (1+\gamma_{\mu})\phi_{x+\mu} \rhodag_x$             \\
\quad  $ds\_u_0 \ += U_{0,\mu,x} utmp\_1 $&                              \\
 & \\
 \quad  Checkerboard 1 & \\
 \quad  $ftmp\_2 \ = -(1-\gamma_{\mu}) \rho $    &                    \\
\quad  $utmp\_1 \ = - ftmp\_2_{x+\mu} \phi^{\dagger}$ &  // $-\kappa^2(1-\gamma_{\mu})\rho_{x+\mu} \phidag_x$       \\
\quad  $ftmp\_2 \ = (1 + \gamma_{\mu}) \sigma$  &                    \\
\quad  $utmp\_1 \ += ftmp\_2_{x+\mu} \psi^{\dagger}$ & // $-\kappa^2 (1+\gamma_{\mu})\sigma_{x+\mu} \psidag_x$             \\
\quad  $ds\_u_1 \ += U_{1,\mu,x} utmp\_1 $&                             \\
\}                         &                                      \\ 
\end{tabular}
}
\caption{Pseudocode for SZIN preconditioned force term (\tt wldsduf\_w.m)  }
\end{center}
\end{figure}



\begin{thebibliography}{99}
\bibitem{Zbysh} 
Z. Sroczynski, Ph.D Thesis, Appendix B.
\end{thebibliography}

\end{document}
