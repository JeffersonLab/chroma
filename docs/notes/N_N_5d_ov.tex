\documentclass[12pt]{article}
\usepackage{amsmath}

% Somewhat wider and taller page than in art12.sty
\topmargin -0.4in  \headsep 0.0in  \textheight 9.0in
\oddsidemargin 0.25in  \evensidemargin 0.25in  \textwidth 6.5in

\footnotesep 14pt
\floatsep 28pt plus 2pt minus 4pt      % Nominal is double what is in art12.sty
\textfloatsep 40pt plus 2pt minus 4pt
\intextsep 28pt plus 4pt minus 4pt

\begin{document}

\def\dirac{{\bf \rm D}\!\!\!\!/\,}
\newcommand{\half}{\frac{1}{2}}
%\newcommand{\be}{\begin{equation}}
%\newcommand{\ee}{\end{equation}}
\newcommand{\be}{\begin{displaymath}}
\newcommand{\ee}{\end{displaymath}}
\newcommand{\bea}{\begin{eqnarray}}
\newcommand{\eea}{\end{eqnarray}}
\newcommand{\bdm}{\begin{displaymath}}
\newcommand{\edm}{\end{displaymath}}
\newcommand{\<}{\langle}
\renewcommand{\>}{\rangle}
\newcommand{\Tr}{\mbox{Tr}}

\centerline{\bf \Large Variations on Narayanan-Neuberger 5-d overlap}
\vskip 5mm

Narayanan and Neuberger's 5-d overlap version (hep-lat/0005004) is based
on the 5-d action
\bea
S^{(5)} = \bar \Psi H^{(5)}_{NN} \Psi
\eea
with the combined ``5-d'' fermion field $\bar \Psi = (\bar \psi, \bar \chi_1,
\bar \phi_1, \dots, \bar \chi_N, \bar \phi_N)$ and similarly for $\Psi$.
Here
\bea
H^{(5)}_{NN} = \begin{pmatrix}
H_{00} & \alpha_1 & 0 & \alpha_2 & 0 & \cdots & \alpha_N & 0 \cr
\alpha_1 & -\beta_1 H_w & \delta_1 & 0 & 0 & \cdots & 0 & 0 \cr
0 & \delta_1 & H_w & 0 & 0 & \cdots & 0 & 0 \cr
\alpha_2 & 0 & 0 & -\beta_2 H_w & \delta_2 & \cdots & 0 & 0 \cr
0 & 0 & 0 & \delta_2 & H_w & \cdots & 0 & 0 \cr
\vdots & \vdots & \vdots & \vdots & \vdots & \vdots & \vdots & \vdots \cr
\alpha_N & 0 & 0 & 0 & 0 & \cdots & -\beta_N H_w & \delta_N \cr
0 & 0 & 0 & 0 & 0 & \cdots & \delta_N & H_w \cr
\end{pmatrix}
\label{eq:H_5d_matrix}
\eea
with as yet unspecified $H_{00}$ and coefficients $\alpha_k$, $\beta_k$
and $\delta_k$ for $k=1, \dots, N$.
The aim is to choose these such that, integrating over the auxiliary
fermion fields $\bar \chi_k$, $\chi_k$, $\bar \phi_k$ and $\phi_k$
one obtains
\bea
\int d \bar \Psi d \Psi {\rm e}^{- \bar \Psi H^{(5)}_{NN} \Psi} =
{\cal N} \int d \bar \psi d \psi {\rm e}^{- \bar \psi H_{eff} \psi}
\eea
where $H_{eff}$ is equal to the hermitian overlap Dirac operator
\bea
H_{ov} = \frac{1}{2} \Bigl[ (1+m) \gamma_5 + (1-m) \epsilon(H_w) \Bigr] ~.
\label{eq:H_ov}
\eea

We collect the terms in $\bar \Psi H^{(5)}_{NN} \Psi$ that depend on the
fermion fields $\bar \chi_k$, $\chi_k$, $\bar \phi_k$ and $\phi_k$
\bea
&& \alpha_k (\bar\psi \chi_k + \bar\chi_k \psi) -
 \bar\chi_k \beta_k H_w \chi_k + \bar\phi_k H_w \phi_k +
 \delta_k (\bar\chi_k \phi_k + \bar\phi_k \chi_k) = \nonumber \\
&& - \Bigl[ \bar\chi_k - \alpha_k \bar\psi (\beta_k H_w +
 \delta_k^2 H_w^{-1} )^{-1} \Bigr] (\beta_k H_w + \delta_k^2 H_w^{-1} )
 \Bigl[ \chi_k - \alpha_k (\beta_k H_w +
 \delta_k^2 H_w^{-1} )^{-1} \psi \Bigr] \nonumber \\
&& + (\bar\phi_k + \delta_k \bar\chi_k H_w^{-1}) H_w
 (\phi_k + \delta_k H_w^{-1} \chi_k)
 + \alpha_k^2 \bar\psi (\beta_k H_w + \delta_k^2 H_w^{-1} )^{-1} \psi ~.
\eea
Doing now the intergration over $\bar \chi_k$, $\chi_k$, $\bar \phi_k$ and
$\phi_k$, after appropriate changes of variables we find a factor
$ \det H_w \det (\beta_k H_w + \delta_k^2 H_w^{-1} ) = \det
(\beta_k H_w^2 + \delta_k^2 )$ in ${\cal N}$ and the contribution
\bea
\alpha_k^2 \bar\psi (\beta_k H_w + \delta_k^2 H_w^{-1} )^{-1} \psi
= \alpha_k^2 \bar\psi H_w (\beta_k H_w^2 + \delta_k^2 )^{-1} \psi
\eea
in the exponent, {\it i.e.,} we find
\bea
H_{eff} = H_{00} + H_w \sum_{k=1}^N \frac{\alpha_k^2}{\beta_k H_w^2 +
 \delta_k^2} ~.
\label{eq:H_eff}
\eea
The second term in (\ref{eq:H_eff}) looks reminiscent of a ``sum over poles''
approximation to $\epsilon(H_w)$
\begin{equation}
g_N(H_w) = H_w \left(c_0 + {\sum_{k=1}^N {c_k \over {H_w^2 + b_k}}}\right),
\label{eq:ratio_pole}
\end{equation}
and suggests the choices
\bea
\beta_k = s^2 ~, \qquad \alpha_k^2 = \frac{1-m}{2} s c_k ~~{\rm and} ~~~
\delta_k^2 = b_k ~.
\label{eq:rat_coeff}
\eea
To make $H_{eff} = H_{ov}$ we further choose
\bea
H_{00} &=& \frac{1+m}{2} \gamma_5 + \frac{1-m}{2} s c_0 H_w + \frac{1-m}{2}
 \sum_i \biggl[ \epsilon(\lambda_i) - s c_0 \lambda_i -
 \sum_{k=1}^N \frac{s c_k \lambda_i}{s^2 \lambda_i^2 + b_k} \biggr] \hat P_i
 \nonumber\\
 &=& \frac{1+m}{2} \gamma_5 + \frac{1-m}{2} s c_0 H_w + \hat{A}(m)
\label{eq:H_00}
\eea
Here, with
\bea
\hat{A}(m) = \frac{1-m}{2}
 \sum_i \biggl[ \epsilon(\lambda_i) - s c_0 \lambda_i -
 \sum_{k=1}^N \frac{s c_k \lambda_i}{s^2 \lambda_i^2 + b_k} \biggr]\hat{P}_i
\eea
and
\bea
H_w v_i = \lambda_i v_i ~, \qquad \hat{P}_i = v_i v_i^\dagger
\eea
the last term implements projection of Wilson eigenvalues for which the
sum-over-poles approximation to $\epsilon(H_w)$ is not sufficiently accurate.
The scale factor $s > 0$ can be used to optimize the approximation over
the rest of the spectrum of $H_w$.

To precondition a 5-d like matrix, we write it as a two by two block matrix
\begin{eqnarray}
A^{(5)} = \begin{pmatrix} {\cal A}_{EE} & {\cal B}_{EO} \\
  {\cal B}_{OE} & {\cal A}_{OO} \end{pmatrix}
\end{eqnarray}
where the upper case characters $E$ and $O$ label even and odd
(checkerboard) sites, respectively, in five dimensions. The matrix
$A^{(5)}$ can be brought into an even-odd block diagonal form using
the lower and upper block triangular matrices
\begin{eqnarray}
L = \begin{pmatrix} I_{EE} & 0 \\ 
{\cal B}_{OE}{\cal A}_{EE}^{-1} & I_{OO} \end{pmatrix},
\qquad
U = \begin{pmatrix} {\cal A}_{EE} & {\cal B}_{EO}\\
0 & I_{OO} \end{pmatrix}
\end{eqnarray}
with
\begin{eqnarray}
L^{-1} = \begin{pmatrix} I_{EE} & 0 \\ 
-{\cal B}_{OE}{\cal A}_{EE}^{-1} & I_{OO} \end{pmatrix},
\qquad
U^{-1} = \begin{pmatrix} {\cal A}_{EE}^{-1} & -{\cal A}_{EE}^{-1}{\cal B}_{EO}\\
0 & I_{OO} \end{pmatrix}
\end{eqnarray}
in the transformation
\begin{eqnarray}
{\tilde A}^{(5)} = L^{-1} A^{(5)} U^{-1} = 
\begin{pmatrix} I_{EE} & 0 \\ 0 &
{\cal A}_{OO} - {\cal B}_{OE}{\cal A}_{EE}^{-1}{\cal B}_{EO}
\end{pmatrix}.
\end{eqnarray}

For this to be a good preconditioning, we want
\begin{eqnarray}
L U = \begin{pmatrix} A_{EE} & {\cal B}_{EO} \\ 
{\cal B}_{OE} & I_{OO} + {\cal B}_{OE}{\cal A}_{EE}^{-1}{\cal B}_{EO}
\end{pmatrix}
\end{eqnarray}
to be a good approximation to $A^{(5)}$. To this end we want 
\begin{eqnarray}
{\cal A}_{OO} \sim I_{OO} + {\cal B}_{OE}{\cal A}_{EE}^{-1}{\cal B}_{EO} .
\label{eq:check}
\end{eqnarray}
This requirement looks most practical when ${\cal A}_{OO}$ and also
${\cal A}_{EE}$ are initially set (close) to the identity. Hence, when
solving 
\begin{eqnarray}
H_{eff}\psi = \gamma_5 b\quad\rightarrow H^{(5)}_{NN}\Psi = 
(\gamma_5 b,0,0,\ldots,0)^{T}
\end{eqnarray}
rescale the system into
\begin{eqnarray}
{\cal S} H^{(5)}_{NN}\Psi = {\cal S} (\gamma_5 b,0,0,\ldots,0)^{T}.
\end{eqnarray}

We consider the case of $H_w$ the hermitian Wilson operator, then
\bea
D_{ee},~ D_{oo} \rightarrow (4-M)~ ,\quad D_{eo} \rightarrow -\half\dirac_{eo}
\eea
where $\dirac$ is the Wilson dslash operator. Here $e$ and $o$ refer
to the even and odd sub-lattice in the 4-d checker-boarding.
We then choose ${\cal A}={\cal S} H^{(5)}_{NN}$ with
\bea
{\cal S} = \begin{pmatrix}
f_0^{-1}\gamma_5 & 0 & 0 & 0 & 0 & \cdots & 0 & 0 \cr
0 & \frac{-\gamma_5}{\beta_1(4-M)} & 0 & 0 & 0 & \cdots & 0 & 0 \cr
0 & 0 & \frac{\gamma_5}{4-M} & 0 & 0 & \cdots & 0 & 0 \cr
0 & 0 & 0 & \frac{-\gamma_5}{\beta_2(4-M)} & 0 & \cdots & 0 & 0 \cr
0 & 0 & 0 & 0 & \frac{\gamma_5}{4-M} & \cdots & 0 & 0 \cr
\vdots & \vdots & \vdots & \vdots & \vdots & \vdots & \vdots & \vdots \cr
0 & 0 & 0 & 0 & 0 & \cdots & \frac{-\gamma_5}{\beta_N(4-M)} & 0 \cr
0 & 0 & 0 & 0 & 0 & \cdots & 0 & \frac{\gamma_5}{4-M}\cr
\end{pmatrix}
\eea
where
\bea
f_0 = \frac{1+m}{2} + \frac{1-m}{2} s (4-M) c_0~.
\eea

The fermion fields are brought into ``kappa''-form, {\it i.e.}, having
1 in the diagonal with $\kappa= 1/{2(4-M)}$, with
\bea
{\cal A}_{EE} = \begin{pmatrix}
h_{ee} & 0 & 0 & 0 & 0 & \cdots & 0 & 0 \cr
0 & 1 & 0 & 0 & 0 & \cdots & 0 & 0 \cr
0 & 0 & 1 & 0 & 0 & \cdots & 0 & 0 \cr
0 & 0 & 0 & 1 & 0 & \cdots & 0 & 0 \cr
0 & 0 & 0 & 0 & 1 & \cdots & 0 & 0 \cr
\vdots & \vdots & \vdots & \vdots & \vdots & \vdots & \vdots & \vdots \cr
0 & 0 & 0 & 0 & 0 & \cdots & 1 & 0 \cr
0 & 0 & 0 & 0 & 0 & \cdots & 0 & 1\cr
\end{pmatrix}
\eea
and
\bea
{\cal B}_{EO} = \begin{pmatrix}
h_{eo} & f_0^{-1}\alpha_1\gamma_5 & 0 & f_0^{-1}\alpha_2\gamma_5 & 0 & \cdots & 
   f_0^{-1}\alpha_N\gamma_5 & 0 \cr
\frac{-\alpha_1\gamma_5}{\beta_1(4-M)}& -\kappa\dirac_{oe} & 
   \frac{-\delta_1\gamma_5}{\beta_1(4-M)} & 0 & 0 & \cdots & 0 & 0 \cr
0 & \frac{\delta_1\gamma_5}{(4-M)} & -\kappa\dirac_{eo} & 0 & 0 & \cdots & 0 & 0 \cr
\frac{-\alpha_2\gamma_5}{\beta_2(4-M)} & 0 & 0 & -\kappa\dirac_{oe} & 
   \frac{-\delta_2\gamma_5}{\beta_2(4-M)} & \cdots & 0 & 0 \cr
0 & 0 & 0 & \frac{\delta_2\gamma_5}{(4-M)} & -\kappa\dirac_{eo} & \cdots & 0 & 0 \cr
\vdots & \vdots & \vdots & \vdots & \vdots & \vdots & \vdots & \vdots \cr
\frac{-\alpha_N\gamma_5}{\beta_N(4-M)} & 0 & 0 & 0 & 0 & \cdots & -\kappa\dirac_{oe} & 
   \frac{-\delta_N\gamma_5}{\beta_N(4-M)} \cr
0 & 0 & 0 & 0 & 0 & \cdots & \frac{\delta_N\gamma_5}{(4-M)} & -\kappa\dirac_{eo} \cr
\end{pmatrix}
\eea
where
\bea
h_{ee} &=& 1 + f_0^{-1}\gamma_5 {\hat A}_{ee}(m)\nonumber\\
h_{eo} &=& -\half f_0^{-1}\frac{1-m}{2} s c_0 \dirac_{eo} + 
   f_0^{-1}\gamma_5\hat{A}_{eo}(m)~.
\eea
The expressions for ${\cal A}_{OO}$ and ${\cal B}_{OE}$ are similar to
the above expressions with $o$ and $e$ interchanged.

For even-odd preconditioning, we need 
\bea
{\cal A}_{EE}^{-1} = \begin{pmatrix}
[h_{ee}]^{-1} & 0 & 0 & 0 & 0 & \cdots & 0 & 0 \cr
0 & 1 & 0 & 0 & 0 & \cdots & 0 & 0 \cr
0 & 0 & 1 & 0 & 0 & \cdots & 0 & 0 \cr
0 & 0 & 0 & 1 & 0 & \cdots & 0 & 0 \cr
0 & 0 & 0 & 0 & 1 & \cdots & 0 & 0 \cr
\vdots & \vdots & \vdots & \vdots & \vdots & \vdots & \vdots & \vdots \cr
0 & 0 & 0 & 0 & 0 & \cdots & 1 & 0 \cr
0 & 0 & 0 & 0 & 0 & \cdots & 0 & 1\cr
\end{pmatrix}.
\eea
It is also convenient to define a 5-d like $\dirac$
\bea
{\cal B}_{EO} = -\kappa \dirac^{(5)}_{EO}
\eea
where
\bea
\dirac^{(5)}_{EO} = \begin{pmatrix}
-\kappa^{-1}h_{eo} & \frac{-\alpha_1}{\kappa f_0}\gamma_5 & 0 & 
   \frac{-\alpha_2}{\kappa f_0}\gamma_5 & 0 & \cdots & 
   \frac{-\alpha_N}{\kappa f_0}\gamma_5 & 0 \cr
\frac{2\alpha_1}{\beta_1}\gamma_5 & \dirac_{oe} & 
   \frac{2\delta_1}{\beta_1}\gamma_5 & 0 & 0 & \cdots & 0 & 0 \cr
0 & -2\delta_1\gamma_5 & \dirac_{eo} & 0 & 0 & \cdots & 0 & 0 \cr
\frac{2\alpha_2}{\beta_2}\gamma_5 & 0 & 0 & \dirac_{oe} & 
   \frac{2\delta_2}{\beta_2}\gamma_5 & \cdots & 0 & 0 \cr
0 & 0 & 0 & -2\delta_2\gamma_5 & \dirac_{eo} & \cdots & 0 & 0 \cr
\vdots & \vdots & \vdots & \vdots & \vdots & \vdots & \vdots & \vdots \cr
\frac{2\alpha_N}{\beta_N}\gamma_5 & 0 & 0 & 0 & 0 & \cdots & \dirac_{oe} & 
   \frac{2\delta_N}{\beta_N}\gamma_5 \cr
0 & 0 & 0 & 0 & 0 & \cdots & -2\delta_N\gamma_5 & \dirac_{eo} \cr
\end{pmatrix}
\eea

For preconditioning to work well, we want the second term in
Eq.~\ref{eq:check} to be small. The major difficulty in the 5-d
approach is the large range of the coefficients $\alpha_i$ and
$\delta_i$. There are not necessarily systematic cancellations so as
to satisfy Eq.~\ref{eq:check}. We can implement an additional
rescaling for this by choosing
\bea
{\tilde A}' = {\cal \Lambda} {\tilde A} {\cal \Lambda}^{-1}.
\eea
with
\bea
{\cal \Lambda} = \begin{pmatrix}
s & 0 & 0 & 0 & 0 & \cdots & 0 & 0 \cr
0 & u_1 & 0 & 0 & 0 & \cdots & 0 & 0 \cr
0 & 0 & v_1 & 0 & 0 & \cdots & 0 & 0 \cr
0 & 0 & 0 & u_2 & 0 & \cdots & 0 & 0 \cr
0 & 0 & 0 & 0 & v_2 & \cdots & 0 & 0 \cr
\vdots & \vdots & \vdots & \vdots & \vdots & \vdots & \vdots & \vdots \cr
0 & 0 & 0 & 0 & 0 & \cdots & u_N & 0 \cr
0 & 0 & 0 & 0 & 0 & \cdots & 0 & v_N\cr
\end{pmatrix}
\eea
This scaling leaves invariant the identity parts of ${\cal A}_{OO}$
and ${\cal A}_{EE}$. We want to use this rescaling to minimize errors
in Eq.~\ref{eq:check}. Consider the case of no projection so ${\cal
A}_{EE}$ is the identity. We can impose a matrix norm for quality of
fit with $|\dirac^{(5)}_{OE}\dirac^{(5)}_{EO}|\equiv\sum_{ijk}
\dirac^{(5)}_{ik}\dirac^{(5)}_{kj}$. This defines a $\chi$-square for
us and we take a variation over $\{s,u_1,v_1,\ldots,u_N,v_N\}$. This
appears complicated to solve, so instead consider the norm
$|\dirac^{(5)}|^p=\sum_{ij}(\dirac^{(5)}_{ik})^p$ for some integer
$p$. Taking the variation and solving the linear system for the
$\{s,u_1,v_1,\ldots,u_N,v_N\}$ we find
\bea
\{s,u_k,v_k\} = \{s,\half\frac{\sqrt{-2\kappa f_0 \beta_k} s}{\kappa f_0},
\half\frac{\sqrt{-\beta_k}\sqrt{-2\kappa f_0 \beta_k} s}{\kappa f_0\beta_k}\}
\eea
independent of $k$ and $p$. Choosing $\beta_k$ to be negative forces
$\alpha_k$ and $\delta_k$ to be pure imaginary which then reduces the
$\dirac$ operator (including projection) to 
\bea
\dirac^{(5)'}_{EO} = \begin{pmatrix}
-\kappa^{-1}h_{eo} & \frac{-\sqrt{2}\alpha_1 i}{\kappa f_0\beta_1}\gamma_5 & 0 & 
   \frac{-\sqrt{2}\alpha_2 i}{\kappa f_0\beta_2}\gamma_5 & 0 & \cdots & 
   \frac{-\sqrt{2}\alpha_N i}{\kappa f_0\beta_N}\gamma_5 & 0 \cr
\frac{-\sqrt{2}\alpha_1 i}{\kappa f_0\beta_1}\gamma_5 & \dirac_{oe} & 
   \frac{2\delta_1 i}{\sqrt{\beta_1}}\gamma_5 & 0 & 0 & \cdots & 0 & 0 \cr
0 & \frac{2\delta_1 i}{\sqrt{\beta_1}}\gamma_5 & \dirac_{eo} & 0 &0 &\cdots & 0 & 0 \cr
\frac{-\sqrt{2}\alpha_2 i}{\sqrt{\beta_2}}\gamma_5 & 0 & 0 & \dirac_{oe} & 
   \frac{2\delta_2 i}{\sqrt{\beta_2}}\gamma_5 & \cdots & 0 & 0 \cr
0 & 0 & 0 & \frac{2\delta_2 i}{\sqrt{\beta_2}}\gamma_5 & \dirac_{eo} &
   \cdots & 0 & 0 \cr
\vdots & \vdots & \vdots & \vdots & \vdots & \vdots & \vdots & \vdots \cr
\frac{-\sqrt{2}\alpha_N i}{\kappa f_0\beta_N}\gamma_5 & 0 & 0 & 0 & 0 &
   \cdots & \dirac_{oe} & \frac{2\delta_N i}{\sqrt{\beta_N}}\gamma_5 \cr
0 & 0 & 0 & 0 & 0 & \cdots & \frac{2\delta_N i}{\sqrt{\beta_N}}\gamma_5 & 
   \dirac_{eo} \cr
\end{pmatrix}
\eea
where $s=1$, $-\beta_k$, $-\alpha_k^2$, and $-\delta_k^2$ have been used.  We
see that in this new form for $D^{(5)'}$ that when comparing
Eq.~\ref{eq:H_eff} to Eq.\ref{eq:ratio_pole} the extra scaling with
$\beta_k$ drops out.

\end{document}
